\section{研究背景}
% 近年深層学習などを用いて、医療分野・製造分野における画像診断、制御分野では自動運転のための画像処理などに用いられているが、深層学習モデルの性能は学習データの量と多様性に強く依存する。特に画像識別タスクでは、学習データが少ないと過学習や汎化性能低下が起こりやすく、モデルの実運用性能を下げる要因になる。この問題に対して、回転・平行移動・反転などの幾何学的な拡張がよく用いられるが、これらは基本的に「既存画像の変形」に留まり、「新規の意味的バリエーション(髪型、服装、形状、テクスチャ等)」を増やすことは難しい。

% 近年、深層学習は医療分野における画像診断や、製造分野における外観検査、さらに自動運転に代表される制御分野の画像処理など、幅広い分野で利用されている。一方で、深層学習モデルの性能は学習データの量および多様性に強く依存することが知られている。特に画像識別タスクにおいては、学習データが不足すると過学習や汎化性能の低下が生じやすく、実運用時の性能劣化を招く要因となる。

% この問題に対する一般的な対策として、回転・平行移動・反転などの幾何学的変換によるデータセット拡張が広く用いられている。しかし、これらの手法は既存画像の変形に留まるため、髪型や服装、形状、テクスチャといった新規の意味的バリエーションを十分に増やすことは困難である。

近年、オンライン空間を活用したサービスが広く普及し、情報発信や交流の場として利用される機会が増加している\cite{soumusho2024_information_communications_whitepaper}。SNS、ゲーム、メタバース、遠隔コミュニケーションなど、物理的な距離に依存しないコミュニケーション手段が社会に定着しつつある。

このようなオンライン空間において、人をどのように表現するかは重要な要素である。現実空間とは異なり、オンライン上では外見や振る舞いを直接的に共有することが難しく、その代替として視覚的な表現手段が用いられてきた。
その代表例として、オンライン空間上で人の存在を表す手段としてアバターが広く利用されている(\fref{fg:abata-})。アバターは単なる装飾ではなく、利用者の個性や属性を反映する表現手段として機能しており、オンライン上でのコミュニケーションにおいて重要な役割を果たしている\cite{yee_proteus_2007}。

本研究では、このようなアバターを対象とし、アバター画像を生成するシステムの設計および実装を行う。具体的には、既存のアバター画像をもとに、新たな外観を持つアバター画像を生成することを目的とする。

また、本研究では画像生成処理をFPGA上に実装することで、生成モデルを用いた画像生成をハードウェアレベルで実現する構成について検討する。

\begin{figure}[htbp]
	\centering
	\includegraphics[width=70mm,keepaspectratio]{figure/abata.png}
	\caption{オンライン空間におけるアバター利用のイメージ図(Geminiを用いて作成)}
	\label{fg:abata-}
\end{figure}




% \subsubsection*{教師あり学習の限界}
% 教師あり学習ではピクセル誤差(L1/L2)を最小化することで画像を学習する。そのため、学習外の入力に対しては、複数の正解候補を同時に満たそうとし、生成結果が画像の平均的な外観となりやすい。特に出力分布が多峰性の場合、この性質は輪郭のぼやけや細部の欠落として顕在化する。したがって、再構成損失中心の生成手法は、データ多様性の拡張という観点では限界がある。

% \subsubsection*{GANの優位性}
% % GANは「特定の入力に対してピクセルを一致させる」のでなく、データ集合全体の分布を再現するように学習することが特徴である。結果として、GANは既存データの単なる変形ではなく、データ分布に整合する新規サンプルを生成できる。したがって、新規の意味的バリエーションをもった画像を生成する上でGANは優位性がある。

% GANは、特定の入力に対してピクセル単位での一致を目指すのではなく、データ集合全体の確率分布を再現するように学習する点に特徴がある。このため、既存データの単なる変形に留まらず、学習データの分布に整合した新規サンプルの生成が可能である。結果として、GANは意味的に多様な画像を生成でき、データセット拡張手法として高い有効性を有する。


% \subsubsection*{本システムの応用例(これでいいのか?)}
% 本システムは、任意のランダム入力に対して、学習データ分布に近い新たなカラー画像を生成できる点に特徴がある。これにより、従来の幾何学的拡張では得られなかった意味的バリエーションを含むデータセット拡張が可能となり、学習モデルの汎化性能向上が期待できる。さらに、本システムはFPGAを用いた高速な画像生成を可能としているため、リアルタイムにデータを取得しながら学習を行うような状況においても有効である。特に、取得可能なデータがスパースな状況において、生成画像によるデータセット拡張をリアルタイムに行うことで、学習データ不足を補完できる可能性がある。