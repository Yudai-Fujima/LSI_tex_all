\section{概要}
今年度のLSIデザインコンテストにおける設計課題は「Generative Adversarial Networks(GAN)」である。GANは、画像生成、超解像、スタイル変換、異常検知などに用いられる生成ネットワークである。我々は、画像生成の応用として、様々な髪型、髪色、目、服のアバターの画像を学習されることで、データセットには含まれない新たなアバターを生成するGANの回路設計および実装を行った。本システムの特徴を以下に挙げる。

\begin{enumerate}
   \item ランダムノイズ(100次元)の入力$z$に対し、新たなアバター画像を生成するGeneratorネットワーク(CNN)のFPGA実装
   \item ランダムノイズ$z$に対し、100次元のランダムベクトル$v1、v2$を用いて、$z$はR画像の学習、$z+v_1$はG画像の学習、$z+v_2$はB画像の学習を行うことで、一つのGeneratorネットワークでカラー画像の出力を実現
   \item ハードの工夫点 並列化など??
   \item ハードの工夫点
\end{enumerate}

2章では、本システムを社会実装するにあたって、データセット拡張における有用性を示す。3章では、準備として生成モデルのなかでのGANの位置づけ、およびGANの学習アルゴリズムについて説明する。4章では、ソフトウェアにおける本システムの特徴を示す。5章では、ハード。