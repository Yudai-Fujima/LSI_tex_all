\section{概要}
今年度のLSIデザインコンテストにおける設計課題は「Generative Adversarial Networks(GAN)」である。GANは、画像生成、超解像、スタイル変換、異常検知などに用いられる生成ネットワークである。我々は、画像生成の応用として、様々な髪型、髪色、目、服のアバターの画像を学習されることで、データセットには含まれない新たなアバターを生成するGANの回路設計および実装を行った。本システムの特徴を以下に挙げる。

\begin{enumerate}
   \item ランダムノイズ(100次元)の入力$z$に対し、新たなアバター画像を生成するGeneratorネットワーク(CNN)のFPGA実装
   \item ランダムノイズ$z$に対し、100次元のランダムベクトル$v1、v2$を用いて、$z$はR画像の学習、$z+v_1$はG画像の学習、$z+v_2$はB画像の学習を行うことで、一つのGeneratorネットワークでカラー画像の出力を実現
   \item 大規模なCNNモデルの実装を目的とした,PSとPLが連携する協調型アーキテクチャ
   \item 今後の高速化、高解像度化が可能な拡張性のある構成
\end{enumerate}

2章では,本システムの設計に取り組む背景を示す。  
3章および4章では,準備として生成モデルの中でのGANの位置づけと,GANの学習アルゴリズムについて説明する。  
5章では,ソフトウェア実装における本システムの特徴を示す。  
6章では,CNNの特徴と本システムでの利用方法について述べる。  
7章では,ハードウェア設計に向けて、エミュレータ側での固定小数点の検討について示す。
そして、8章では提案するシステム構築に用いる使用機器についてまとめる。
9章では、ハードウェア構成(PS/PL協調・AXI通信)を説明し、 
10章では,Generatorの実装についてまとめる。
11章では、FPGA実装にあたっての工夫点をまとめている。
12章、13章ではそれぞれシミュレーション動作、実機動作について記載し、14章で本システムの今後の展望をまとめた。
15章でまとめを、16章で謝辞を述べる。