\section{概要}
今年度のLSIデザインコンテストにおける設計課題は「Generative Adversarial Networks(GAN)」である。GANは、画像生成、超解像、スタイル変換、異常検知などに用いられる生成ネットワークである。我々は、画像生成の応用として、様々な髪型、髪色、目、服のアバターの画像を学習されることで、データセットには含まれない新たなアバターを生成するGANの回路設計および実装を行った。

% 本システムを実現するにあたり用いた、学習モデルの概念図を\ref{fg:1_soft}に示す。モデル中のGeneratorが画像生成ネットワークであり、4層の逆畳み込み層から構成される。入力は正規分布を基に生成された100次元の潜在ベクトルであり、出力は32×32の画像である。学習後、潜在ベクトルは画像の特徴を捉えたベクトルとして機能する。
本システムの実現のために構成した学習モデルの概念図を図 \ref{fg:1_soft}に示す。本モデルにおける Generator は画像生成ネットワークであり、4層の逆畳み込み層から構成される。入力は100次元の潜在ベクトル、出力は32×32画素の画像である。学習後、潜在ベクトルは画像の特徴を表現するベクトルとして機能する。
\begin{figure}[htbp]
	\centering
	\includegraphics[width=70mm,keepaspectratio]{figure/1_soft.png}
	\caption{アバター作成のためのGANモデル\\   (学習イメージ)}
	\label{fg:1_soft}
\end{figure}

今回はGANモデルのうちGenerator部分のみをFPGA に実装した。FPGA上に構成するGeneratorは、4層の逆畳み込み層からなる。学習済みGeneratorのパラメータをFPGA上の回路に組み込むことにより、利用者は学習時と同様の分布に基づく新規の潜在ベクトルを入力するだけで、これまでに存在しない新たなアバター画像を生成できる。(\fref{fg:1_fpga})
\begin{figure}[htbp]
	\centering
	\includegraphics[width=70mm,keepaspectratio]{figure/1_fpga.png}
	\caption{FPGA実装部分の概念図\\  (システム利用時の想定図)}
	\label{fg:1_fpga}
\end{figure}


以下に、本レポートの目標および本システムの工夫点をまとめる。
\subsubsection*{目標}
\textbf{「潜在ベクトル $z$(100次元)の入力に対し、新たなアバター画像を生成する Generator ネットワーク(4層の逆畳み込み層)の FPGA 実装」}

\subsubsection*{工夫点(ソフトウェア)}
\begin{enumerate}
   \item  一つの潜在ベクトル$z$の入力に対して、R・G・Bの3種類の画像を出力する学習モデルの構築\\(学習モデルは第5章、理論は補足資料1)
   \item 二種類のDiscriminatorネットワークを用いた、カラー画像生成の精度向上
   \item 固定小数点を意識したエミュレータ設計
\end{enumerate}
\subsubsection*{工夫点(ハードウェア)}
\begin{enumerate}
   \item 大規模なCNNモデルの実装を目的とした,PSとPLが連携する協調型アーキテクチャ
   \item 今後の高速化、高解像度化が可能な拡張性のある構成
\end{enumerate}

2章では,本システムの設計に取り組む背景を示す。  
3章および4章では,準備として生成モデルの中でのGANの位置づけと,GANの学習アルゴリズムについて説明する。  
5章では,ソフトウェア実装における本システムの特徴を示す。  
6章では,CNNの特徴と本システムでの利用方法について述べる。  
7章では,ハードウェア設計に向けて、エミュレータ側での固定小数点の検討について示す。
そして、8章では提案するシステム構築に用いる使用機器についてまとめる。
9章では、ハードウェア構成(PS/PL協調・AXI通信)を説明し、 
10章では,Generatorの実装についてまとめる。
11章では、FPGA実装にあたっての工夫点をまとめている。
12章、13章ではそれぞれシミュレーション動作、実機動作について記載し、14章で本システムの今後の展望をまとめた。
15章でまとめを、16章で謝辞を述べる。