\newpage

\section{エミュレータ設計}
前章までの学習では、PyTorch などの既存ライブラリを用いており、逆畳み込み演算には \texttt{nn.ConvTranspose2d} を使用していた。そのため、具体的な逆畳み込みの計算は関数で処理されている上、計算はすべて浮動小数点演算により行われている。

本章では、FPGA実装にあたり、実際にエミュレータに構成した、逆畳み込み演算の具体的な計算の流れを整理するとともに、固定小数点表現の検討を行う。


\subsection{Generatorの逆畳み込み回路}
本システムの逆畳み込み回路は、\fref{fg:CNN7}のような構成をしており、4×4のサイズのカーネルに対して、各パラメータに基づいた演算を行うことにより、1×1→4×4→8×8→16×16→32×32とスケールアップしていく。

\begin{figure}[htbp]
	\centering
	\includegraphics[width=60mm,keepaspectratio]{figure/CNN7.png}
	\caption{逆畳み込み回路の入出力関係}
	\label{fg:CNN7}
\end{figure}

\subsubsection*{一つのカーネルと特徴マップの逆畳み込み計算}
例として、Deconv2における4×4→8×8の計算を\fref{fg:CNN8}に示す。カーネルサイズは4×4である。まず、stride=2より\fref{fg:CNN4}のデータ拡張を行うことで7×7になる。次に、STEP2の余白の追加は「カーネル数$-1=3$」の0が追加される。STEP3ではpadding=1より、1行分の0が削除され、11×11となる。STEP4で通常の畳み込み演算を行うことにより、8×8の特徴マップが生成できる。

\begin{figure}[htbp]
	\centering
	\includegraphics[width=70mm,keepaspectratio]{figure/CNN8.png}
	\caption{Deconv2における4×4→8×8の計算例}
	\label{fg:CNN8}
\end{figure}

\subsubsection*{一層ごとの逆畳み込み演算}
本節では、一層ごとの逆畳み込み層ではどのような演算が行われているかを説明する。本システムにおける各層ごとの入出力およびカーネル行列の関係を\fref{fg:CNN9}に示す。

\begin{figure}[htbp]
	\centering
	\includegraphics[width=70mm,keepaspectratio]{figure/CNN9.png}
	\caption{各層ごとの入出力およびカーネル行列の関係}
	\label{fg:CNN9}
\end{figure}

\fref{fg:CNN9}の計算関係から分かるように、各層では入力input$[1][i]$とカーネルkernel$[i][k]$の行列積を計算することで、出力output$[k]$を得ている。$j$番目の出力を得るには、$j$列のカーネルkernel$[i][j]$と入力input$[1][i]$の行列積をとる。つまり、出力の一要素を計算したい場合は、カーネルは該当する一列分の要素のみが必要となる。具体的な計算を次節で説明する。

\subsubsection*{Deconv1の逆畳み込み演算例}
本節では、Deconv1の逆畳み込み演算を例として、各層の演算の流れを示す。Deconv1では、100次元の入力input$[1][i](i=1,2,\cdots,100)$に対して、512次元のoutput$[k] (k=1,2,\cdots,512)$を出力する。各outputに対する計算は列ごとに行うため、1列目の計算を例にする。その概要図を\fref{fg:CNN10}に示す。

\begin{figure}[htbp]
	\centering
	\includegraphics[width=60mm,keepaspectratio]{figure/CNN10.png}
	\caption{Deconv1の1列目の計算例}
	\label{fg:CNN10}
\end{figure}

\fref{fg:CNN10}のように、一列目の計算は\eref{eq:1retu}となる。
\begin{align}
	\Sigma_{i=1}^{100} \text{input1}[1][i]\cdot\text{kernel1}[i][1]=\text{output1}[1] \label{eq:1retu}
\end{align}

これで1列分の計算が完了するため、同様の計算を残りの511列に対して行う(\fref{fg:CNN11})。つまり、Deconv1層における畳み込み演算は\eref{eq:zenretu}となる。
\begin{align}
	&\Sigma_{k=1}^{512}\left[ \Sigma_{i=1}^{100} \text{input1}[1][i]\cdot\text{kernel1}[i][k]\right] \notag\\
	&=\text{output1}[k]\label{eq:zenretu}
\end{align}

\begin{figure}[htbp]
	\centering
	\includegraphics[width=60mm,keepaspectratio]{figure/CNN11.png}
	\caption{Deconv1の全列の計算例}
	\label{fg:CNN11}
\end{figure}