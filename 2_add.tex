\section{画像生成手法とGANの位置づけ}
アバター画像を生成するにあたり、どのような画像生成手法を用いるかは重要である。以下では、従来の画像生成手法の特徴を簡潔に整理し、本研究で用いる手法の位置づけを示す。

\subsection{教師あり学習による画像生成}
教師あり学習に基づく画像生成では、入力画像と対応する正解画像の組を用いて学習を行い、生成画像と正解画像との差を損失関数として最小化する。このような手法では、画素ごとの差分を評価する損失関数が用いられることが多く、代表的なものとして二乗誤差や絶対誤差が挙げられる。

これらの手法は、入力と出力の対応関係が明確なタスクにおいて有効であり、既存画像の再構成や変換といった用途で広く用いられてきた。

\subsection{教師あり学習の限界}
画像生成タスクにおいて、出力が多様な分布を持つ場合、画素単位の誤差を最小化する教師あり学習では、複数の正解を同時に満たそうとする結果、生成画像が平均的な外観となる傾向がある。この性質は、生成結果において輪郭のぼやけや細部表現の欠落として現れることがある。

また、教師あり学習では学習データに含まれない特徴を持つ画像を生成することが難しく、生成結果が既存データの補間にとどまる場合が多い。そのため、外観の多様性を持つ画像を生成する用途においては、別のアプローチが検討されている。

\subsection{GANの特徴と優位性}
GANは、生成モデルと識別モデルを用いた敵対的学習によって画像生成を行う手法である。生成モデルはランダムな入力から画像を生成し、識別モデルは生成画像と実画像とを判別する。両者を同時に学習させることで、生成モデルは実画像の分布に近い画像を生成するように更新される。

GANの特徴は、特定の正解画像との画素単位での一致を目的とせず、画像全体の分布を学習する点にある。このため、学習データの特徴を反映した新たな画像を生成でき、教師あり学習に基づく手法と比べて、より自然な外観を持つ画像の生成が可能である。