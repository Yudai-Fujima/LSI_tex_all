\section{教師あり学習と GAN による画像生成}
本章では、画像生成に用いられる代表的な学習手法として、教師あり学習およびGANによる画像生成の考え方を整理する。

\subsection{教師あり学習による画像生成}
教師あり学習よる画像生成では、入力データと対応する正解画像の組を用いて学習が行われる。学習過程においては、生成画像と正解画像との差を損失関数として定義し、その値が最小となるようにモデルのパラメータが更新される。このとき、画素ごとの値の差分を評価する損失関数が用いられることが多く、モデルは正解画像の画素値を再現する方向に学習が進む。

このような学習方法は、入力と出力の対応関係が明確に定義されているタスクにおいて有効であり、既存画像の再構成などの用途で広く用いられてきた。学習データに含まれる入力に対しては、正解画像に近い出力を安定して得ることができる点が特徴である。

一方で、教師あり学習では学習時に与えられた正解画像を基準としてパラメータが更新されるため、学習データに含まれない新規の入力に対しては、明確な正解画像が存在しない。この場合、モデルは複数の妥当な出力を同時に満たそうとする挙動となり、結果として生成画像が平均的な外観となることがある。

\subsection{GAN による画像生成}
GANでは、生成器と識別器の二つのネットワークを用いた学習が行われる。生成器は入力から画像を生成し、識別器はその画像が正解データに由来するものかどうかを判別する。学習過程において、生成器は識別器の判別結果をもとにパラメータを更新する。

この学習では、生成画像が特定の正解画像と画素ごとに一致しているかどうかではなく、正解データとして与えられた画像群に含まれる特徴を持っているかどうかが評価基準となる。したがって、生成器は正解画像そのものを再現するのではなく、正解データに共通する外観の特徴を反映した画像の生成方法を学習する。

このため、新規の入力が与えられた場合においても、生成器は正解データらしい外観を持つ画像を出力することが可能となる。教師あり学習とGANは、いずれも画像生成に用いられる手法であるが、学習の目的および評価の基準が異なっており、それぞれ異なる特性を持つ手法として位置づけられる。